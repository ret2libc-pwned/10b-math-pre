\documentclass{beamer}

% If you use the optional handout parameter below, none of the pauses between slides are printed.  Only one of these first two lines should be used at a time.
% \documentclass[handout]{beamer}

\setbeamertemplate{navigation symbols}{}

\usepackage{amsmath, amsthm, amssymb, upgreek, amscd, verbatim, xcolor}
\usepackage{centernot}
\usepackage{datetime}
\usepackage{fontspec}
\usepackage[slantfont, boldfont]{xeCJK}
\usepackage{graphicx} 
\usepackage{float} 
\usepackage{subfigure}
\usepackage{tikz}
\usepackage{unicode-math} 
\usepackage{extarrows}

\usetheme{Madrid}
\usecolortheme{seahorse}
\setbeamertemplate{items}[circle]
\usepackage{graphicx}

% Use the XITS font
\setmathfont{XITSMath-Regular}
\setmainfont{XITSMath-Regular}
\setsansfont{XITSMath-Regular}
\setmonofont{XITSMath-Regular}

\setCJKmainfont{SimSun}[BoldFont=SimHei, ItalicFont=KaiTi]

% File matadata
\newcommand{\DocumentTitle}{浅谈递推关系的常见求解方法}
\newcommand{\DocumentAuthor}{胡顾笑、张溥原、欧嘉欢、朱师彤}

\hypersetup{
pdftitle={\DocumentTitle},
% pdfsubject={lorem ipsum},  % TODO
pdfauthor={\DocumentAuthor}, 
% pdfkeywords={lorem ipsum}  % TODO
}

%New Commands
\newcommand{\ssm}{\,{}^\smallsetminus\,}
\newcommand{\R}{\mathbb{R}}
\newcommand{\Q}{\mathbb{Q}}
\newcommand{\Z}{\mathbb{Z}}
\newcommand{\N}{\mathbb{N}}
\newcommand{\F}{\mathbb{F}}

%Date
\newdateformat{mydate}{\THEDAY \ \monthname[\THEMONTH] \THEYEAR}

\begin{document}
\title{\textbf\DocumentTitle}
\author[国际课程班高一下数学小组展示]{\DocumentAuthor}
\date{\mydate\today}

\maketitle

\section{数列与递推关系简介}

\begin{frame}{递推关系}
    在本学期“数列”章节中,我们学到了使用\textbf{递推关系}描述的数列。
    \begin{definition}
        \textbf{递推关系}是一种递推地定义一个序列的方程式:序列的每一项是定义为前若干项的函数。它是一种\textbf{差分方程}。
    \end{definition}
    \begin{example}[Fibonacci 数列的递推形式]
        \begin{displaymath}
            F_i = \begin{cases} 
                1 & i=0,1\\
                F_{i-1} + F_{i-2} & i\geqslant 2
              \end{cases}
        \end{displaymath}

        根据递推式,可以得出其前几项为
        $$
        \{1,1,2,3,5,8,13,\cdots\}
        $$
    \end{example}
\end{frame}

\begin{frame}{递推关系的解}
    \begin{definition}
        求一个递推关系的关于下标 $i$ 的非递推函数的过程,叫做求这个递推关系的\textbf{解析解},也叫求这个递推序列的\textbf{解}。
    \end{definition}
    \begin{example}[Fibonacci 数列的解析解]
        我们熟知,
        $$
            F_i = \frac{1}{\sqrt 5} \phi^i - \frac{1}{\sqrt 5} {\hat\phi}^i = \left\lfloor\frac12+\frac{1}{\sqrt 5}\phi^i\right\rfloor
        $$
        (其中,$\phi$、$\hat\phi$ 是方程 $x^2 - x - 1 = 0$ 的两根,$\phi \xlongequal{\text{def}} \frac{1 + \sqrt 5}{2}\approx 1.618$)
        % (其中,$\phi \xlongequal{\text{def}} \frac{1 + \sqrt 5}{2} \approx 1.618$。$\hat\phi$ 是 $\phi$ 的\textbf{共轭}。下同)
    \end{example}
\end{frame}

\begin{frame}{引入}
    \begin{itemize}
        \item 我们已经学过求解递推关系的基本方法:先猜出递推关系的解,再施\textbf{数学归纳法}证明。
        \item 对于一些特殊的递推关系,有一些方法可以简便地求解。
    \end{itemize}
\end{frame}

\section{从生成函数到特征方程}

\begin{frame}{生成函数}
    我们希望将递推关系描述成我们熟知的多项式函数,或\textbf{幂级数},同时用一种简洁的方式保留这个数列的所有信息。因此引入\textbf{生成函数}的概念。
    \begin{definition}
        定义无穷数列 $\{a_\infty\}$ 的\textbf{生成函数}为:
        $$
            g(x) = \sum_{i=0}^\infty a_ix^i
        $$
    \end{definition}
    \pause
    \begin{itemize}
        \item 生成函数实际上是将这个数列映射到一个 $x$ 进制数。
        \item 作为如此熟悉的代数对象,生成函数自然很适合使用我们熟悉的代数方法处理。例如,$g(x)$ 任意阶可导。
    \end{itemize}
\end{frame}

\begin{frame}{特征方程}
    特征方程是处理常系数线性齐次递推式的方法。

    得到生成函数的启发,我们希望能简单地将数列 $\{a_\infty\}$ 的通项公式表达为 $a_i = q^i$。
    \pause
    \begin{block}{Exercise}
        可以使用生成函数的方法证明其正确性,留作思考。
    \end{block}
\end{frame}

\begin{frame}{特征方程}
    % 我们不加证明地给出以下定理:
    \begin{theorem}[特征方程与特征根]
        设 $q \ne 0$,则 $h_n = q^n$ 为以下常系数 ($\{a_n\}$) 线性齐次递推式
        $$
            h_n - a_1h^{n-1} - a_2h^{n-2} - \cdots - a_kh^{n-k} = 0 
        $$
        的解,当且仅当 $q$ 是方程
        $$
            x^k - a_1x^{k-1}-a_2x^{k-2}-\cdots-a_k = 0
        $$
        的根。称上式为 $\{h_n\}$ 的\textbf{特征方程},该方程的根称为 $\{h_n\}$ 的\textbf{特征根}。
    \end{theorem}
\end{frame}

\begin{frame}{特征方程}
    \begin{theorem}[常系数线性非齐次递推关系式的通解]
        根据\textbf{代数基本定理},$k$ 阶常系数线性齐次递推关系在 $\mathbb C$ 上有 $k$ 个特征根 (含重根)。若有 $m \le k$ 个相异的复根,则
        $$
            h_n = \sum_{i=1}^m C_iq_i^n
        $$
        为这个常系数线性齐次递推关系的\textbf{通解}。
    \end{theorem}
\end{frame}

\section{求解常系数线性齐次递推关系式}

\begin{frame}{常系数线性齐次递推关系式}
    \begin{definition}
        若数列 $\{f_\infty\}$ 的递推关系式满足:
        \begin{itemize}
            \item 线性。$\deg_{f_j} (f_i) \leqslant 1$ ($j < i$)。(但系数与常数不一定需要关于 $i$ 线性)
            \item 齐次。关系式的常数项为 0。
        \end{itemize}
        则称其为\textbf{常系数线性齐次递推关系式}。
    \end{definition}    
\end{frame}

\begin{frame}{例题:广义 Fibonacci 数列}
    \begin{definition}
        一种\textbf{广义 Fibonacci 数列}由如下递推式定义:
        \begin{displaymath}
            F_i = \begin{cases} 
                A & i=0\\
                B & i=1\\
                \alpha F_{i-1} + \beta F_{i-2} & i\geqslant 2
              \end{cases}
        \end{displaymath}
        (其中 $A$、$B$ 为常数)
        
        尝试求解这个递推式。
    \end{definition}    
\end{frame}

\begin{frame}{例题:广义 Fibonacci 数列}
    \begin{solution}
        数列的特征方程为
        $$
        x^2 - \alpha x - \beta = 0 \Longrightarrow x_{1,2} = \frac{\alpha \pm \sqrt{\alpha^2 + 4\beta}}{2}
        $$
        \pause
        因此,该递推式解为
        $$
        F_i = C_1 \left(\frac{\alpha + \sqrt{\alpha^2 + 4\beta}}{2}\right)^i + C_2 \left(\frac{\alpha - \sqrt{\alpha^2 + 4\beta}}{2}\right)^i
        $$
        验证这个解。令 $A = B = \alpha = \beta = 1$,递推式的解变成了\textbf{斐波那契数列}的通项公式。
    \end{solution}

\end{frame}

\begin{frame}{思考题}
    作为习题,请思考如何使用生成函数法求解斐波那契数列的递推式。(提示:先解出生成函数 $g(x)$,再使用一些代数技巧化成通项公式的形式)
\end{frame}

\section{求解常系数线性非齐次递推关系式}

\begin{frame}{常系数线性非齐次递推关系式}
    若常系数线性递推关系式中存在常数项,则称其为\textbf{常系数线性非齐次递推关系式}。
    \pause
    \begin{example}[Hanoi 问题]
        求解递推式
        \begin{displaymath}
            H_i = \begin{cases} 
                0 & i=0\\
                2H_{i-1} + 1 & i\geqslant 1
              \end{cases}
        \end{displaymath}
        ($H_n$ 表示求解 $n$ 层\textbf{汉诺塔}游戏的最小移动步数)
    \end{example}
    \pause
    \begin{block}{Exercise (期末考试押题)}
        尝试猜出这个递推式的解并施数学归纳法证明。
    \end{block}
\end{frame}

\begin{frame}{例题:Hanoi 问题}
    \begin{solution}
        令 
        $$
        g(x) = \sum_{i=0}^\infty H_ix^i
        $$ 
        为 $\{H_\infty\}$ 的生成函数。
        \pause
        将两边同时乘以 $2x$,得到
        $$
        2xg(x) = \sum_{i=0}^\infty 2H_ix^{i+1}
        $$
        \pause
        两式相减,回顾 $H_i$ 的递推定义,解得
        $$
            g(x) = \frac{\sum_{i=1}^\infty x^i}{1-2x}
        $$
    \end{solution}
\end{frame}

\begin{frame}{Hanoi 问题}
    \begin{solution}
        \begin{displaymath}
            \begin{aligned}
                g(x) &= \frac{\sum_{i=1}^\infty x^i}{1-2x} \\
                &= \frac{x}{(1-x)(1-2x)}\\
                &= \frac{1}{1-2x} - \frac{1}{1-x}\\
                &= \sum_{i=1}^\infty (2x)^i - \sum_{i=1}^\infty x^i\\
                &= \sum_{i=1}^\infty (2^i - 1)x^i
            \end{aligned}
        \end{displaymath}
        因此,$\{H_\infty\}$ 的通项公式为
        $$
            H_i = 2^i - 1
        $$
    \end{solution}
\end{frame}

\section{拓展与思考}

\begin{frame}{彩蛋:其它视角下的递推问题}
    以下问题对于解决各种递推问题有一定帮助,但超出了本次展示讨论的范围,有兴趣可自行了解。
    \pause
    \begin{itemize}
        \item 有的时候,我们不需要得出递推关系的解就能足够快地求出某项的信息。
        
                一些计算机课程会介绍几种技巧加速递推的过程。例如对于常系数线性齐次递推常使用矩阵乘法,利用幂运算的结合律使用\textbf{平方法}简化运算数量。
        \begin{displaymath}
            \begin{bmatrix}
                f_n & f_{n-1} & \cdots & f_{n-k}
               \end{bmatrix} = \begin{bmatrix}
                f_k & f_{k-1} & \cdots & f_{0}
               \end{bmatrix} \times (\boldsymbol{A}_{k\times k})^n
        \end{displaymath}
        \pause
        \item 在展示的开头,我们强调过递推关系是一种差分方程。精通微积分的同学可以尝试将 $n$ 阶递推关系看作 $n$ 阶微分方程。事实上,两者的本质几乎相同。
        
            有兴趣的同学可以研究微积分中的工具在求解递推式时的应用,例如\textbf{欧拉法 (Euler's method)}。
    \end{itemize}
\end{frame}

\section{致谢}

\begin{frame}{Thanks}
    \begin{center}
        \Huge{谢谢!}
    \end{center}
\end{frame}
\end{document}